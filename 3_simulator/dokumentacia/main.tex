\documentclass[11pt]{article} % Font size
%\usepackage[utf8]{inputenc}
%\usepackage{graphicx}
%\graphicspath{{Figures/}{./}}
%\usepackage[slovak]{babel}
%\usepackage{amsmath}
\input{structure.tex} % Include the file specifying the document structure and custom commands

%----------------------------------------------------------------------------------------
%	TITLE SECTION
%----------------------------------------------------------------------------------------

\title{	
	\normalfont\normalsize
	\textsc{Slovenská technická univerzita v Bratislave, fakulta elektrotechniky a informatiky}\\ % Your university, school and/or department name(s)
	\vspace{25pt} % Whitespace
	%\rule{\linewidth}{0.5pt}\\ % Thin top horizontal rule
	\vspace{20pt} % Whitespace
	\vspace{20pt} % Whitespace
	\vspace{20pt} % Whitespace
	\vspace{20pt} % Whitespace
	\vspace{20pt} % Whitespace
	\vspace{20pt} % Whitespace
	\vspace{20pt} % Whitespace
	\vspace{20pt} % Whitespace
		\vspace{20pt} % Whitespace
	\vspace{20pt} % Whitespace
	\huge Simulátor diabetu\\ % The assignment title
	\vspace{12pt} % Whitespace
	\normalsize Biokybernetika, zadanie č.3
	\vspace{20pt} % Whitespace
	\vspace{20pt} % Whitespace
	\vspace{20pt} % Whitespace
		\vspace{20pt} % Whitespace
	\vspace{20pt} % Whitespace
		\vspace{20pt} % Whitespace
	\vspace{20pt} % Whitespace
		\vspace{20pt} % Whitespace
	\vspace{20pt} % Whitespace
	\vspace{20pt} % Whitespace
	\vspace{20pt} % Whitespace
		\vspace{20pt} % Whitespace
	\vspace{20pt} % Whitespace
	%\rule{\linewidth}{2pt}\\ % Thick bottom horizontal rule
	\vspace{12pt} % Whitespace
}

\author{\LARGE Lukáš Šníder} % Your name

\date{október 2020}
%\date{\normalsize\today} 

\begin{document}

\maketitle % Print the title
\thispagestyle{empty}
\clearpage
\setcounter{page}{1}
\pagenumbering{arabic} 
%----------------------------------------------------------------------------------------
%	FIGURE EXAMPLE
%----------------------------------------------------------------------------------------

\section{Zostavenie simulačnej schémy podsystému pre }

\subsection{Opis modelu}

%Tento podsystém je reprezentovaný nasledovnými rovnicami:
%%\begin{align*}
%\begin{eqnarray}
%	\dot{S_1}(t) &=& -\left(\frac{1}{T_I}\right) S_1(t) + v(t) \\ 
%	\dot{S_2}(t) &=& -\left(\frac{1}{T_I}\right)S_2(t) + \left(\frac{1}{T_I}\right)S_1(t) \\
%	\dot{I}(t) &=& -{k_I}I(t) + \left(\frac{1}{T_I}\right)\left(\frac{1}{V_I}\right)S_2(t) 
%\end{eqnarray}
%
%Vstupom je rýchlosť podávania inzulínu \textit{v(t)} $[\mu U/kg/min]$ do podkožia, parametrami sú: \textbf{$T_I$ $[min]$}, čo je časová konštanta podsystému, \textbf{$k_I$ $[l/min]$} je rýchlosť samovoľného ubúdania inzulínu a \textbf{$V_I$ $[ml/kg]$} je objem na kilogram hmotnosti. Výstupom je koncentrácia inzulínu v krvi \textit{I(t)}.

\subsection{Zobrazenie dát}

\end{document}
